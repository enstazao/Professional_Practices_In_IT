\documentclass{article}
\usepackage{graphicx}
\usepackage{tabularx}
\usepackage{booktabs}
\usepackage{float}
\usepackage{enumitem}



\title{Comparative Analysis: Elections, Lawmaking, and Checks in USA and Pakistan}
\author{Jawad Ahmed(20P-0165-BCS-7A)}
\date{\today}

\begin{document}

\maketitle

\section{Introduction}

In this report, we will explore and compare the electoral processes, lawmaking procedures, and the implementation of checks and balances in the United States and Pakistan.

\section{Elections in the USA}
In the United States, presidential elections are held to choose the country's leader.
Let's understand the Elections in the USA with an example:
\\
Let's say two major political parties, the Democrats and the Republicans, each nominate a candidate to run for president. For instance, Joe Biden might represent the Democratic Party, and Donald Trump the Republican Party. But here's the twist: the U.S. doesn't rely solely on the popular vote to decide the winner. Instead, they use a system called the Electoral College. Each state has a group of electors, and the candidate who wins in a state gets all its electoral votes. The number of electors each state gets depends on how many members of Congress it has.  There are 538 electoral votes in total, and to become President, a candidate needs at least 270 of them. 
\\
So it's an electoral college that matters not winning the popular vote and 2016 is a good example of that.  In 2016, Donald Trump won crucial states by a slim margin, gaining all their electoral votes, even though Hillary Clinton had more overall votes. The results are shown in Figure 1.

\begin{figure}
  \centering
  \includegraphics[width=0.5\textwidth]{pit-assign-01.png}
  \caption{Elections in USA}
  \label{fig:elections-in-usa}
\end{figure}



\section{Differences of Elections in Pakistan and USA}
\begin{table}[H] % Use [H] to enforce exact placement
\centering
\caption{Key Differences of Elections in usa and Pakistan}
\begin{tabularx}{\textwidth}{X|X}
\toprule
\textbf{United States} & \textbf{Pakistan} \\
\midrule
Electoral System & Parliamentary System \\
Direct Election of President & Direct Election of National Assembly and Senate Members \\
Electoral Votes per State & Proportional Representation in the National Assembly \\
Legislative Structure: House of Representatives and Senate & Legislative Structure: National Assembly and Senate \\
4-Year Presidential Term & 5-Year Prime Ministerial Term \\
Impeachment for President Removal & Vote of No Confidence for Prime Minister Removal \\
\bottomrule
\end{tabularx}
\end{table}

\section{Lawmaking in the USA}
In the United States, the lawmaking process involves several steps:
\begin{enumerate}[label=\arabic*.]
    \item \textbf{Bill Drafting:} Any member of the House of Representatives or the Senate can draft a bill.
    
    \item \textbf{Committee Review:} The bill is then subject to review by one or more committees in the chamber where it was introduced. These committees hold debates, seek advice from stakeholders and experts, and may make amendments to the bill.
    
    \item \textbf{Floor Debate and Vote:} If the bill passes the committee review, it proceeds to the full chamber (House or Senate) for debate and a vote. Members of the chamber have the opportunity to propose amendments, and the bill requires a majority vote for passage.
    
    \item \textbf{Presidential Approval:} Finally, if the bill is approved by both chambers, it is sent to the President. The President can choose to sign the bill into law, veto it, or take no action. In the case of a veto, Congress can override it with a two-thirds majority vote in both chambers.
\end{enumerate}

The concept of Lawmaking in the USA is shown in Figure 2.
\begin{figure}
  \centering
  \includegraphics[width=0.5\textwidth]{law-making-in-usa.jpg}
  \caption{Law Making in the USA}
  \label{fig:law-making-in-usa}
\end{figure}



\section{Law Making in Pakistan}
In Pakistan, the Law making process involves the following key steps:


\begin{enumerate}[label=\arabic*.]
    \item \textbf{Bill Drafting:} A bill is drafted by a member of the National Assembly or the Senate.
    
    \item \textbf{First Reading:} The bill is introduced in the house where it was proposed (National Assembly or Senate).
    
    \item \textbf{Committee Review:} The bill is referred to a relevant committee for review, including debates and input from stakeholders and experts.
    
    \item \textbf{Second Reading and Debate:} Following committee review, the bill proceeds to the second reading and debate, where amendments may be proposed.
    
    \item \textbf{Vote:} A majority vote in the house is required for the bill to pass.
    
    \item \textbf{Consideration in Other House:} If passed, the bill goes to the other house for a similar process.
    
    \item \textbf{Presidential Approval:} After approval by both houses, the bill is sent to the President for final approval or veto.
    
    \item \textbf{Publication:} If approved, the law is published in the Gazette of Pakistan.
\end{enumerate}


\section{Key Differences in Law Making of Pakistan and USA}

\begin{table}[H]
\centering
\caption{Key Differences in Lawmaking Process}
\begin{tabularx}{\textwidth}{|X|X|}
\toprule
\textbf{Pakistan} & \textbf{United States} \\
\midrule
Pakistan has a bicameral legislature with the National Assembly having more power than the Senate & The United States also has a bicameral legislature with equal power between the House of Representatives and the Senate. \\

Pakistani lawmaking is more centralized with power concentrated in the federal government. & US lawmaking is more decentralized with power devolved to state and local governments. \\

Islam is the state religion, and all laws must align with Islamic law. & No official state religion; constitution upholds the separation of church and state. \\

\bottomrule
\end{tabularx}
\end{table}


\section{Concept of Check and Balance in the USA}
The concept of checks and balances is designed to ensure that power in a government is divided, preventing any single branch from becoming too powerful.

In the United States, this concept is put into practice by dividing the government's authority into three main branches:

\begin{enumerate}
  \item The Legislative Branch, is responsible for creating laws.
  \item The Executive Branch, is in charge of enforcing these laws.
  \item The Judicial Branch, interprets the laws and ensures they align with the U.S. Constitution.
\end{enumerate}

This system ensures that each branch of government has its own distinct role and can oversee the actions of the others. It prevents any one branch from gaining too much power and helps maintain accountability in the government. In this way, the concept of checks and balances is effectively applied in the United States.
The concept of check and balance is shown in Figure 3.
\begin{figure}
  \centering
  \includegraphics[width=0.5\textwidth]{check-and-balance-in-us.png}
  \caption{Check and Balance in the USA}
  \label{fig:check-and-balance-usa}
\end{figure}

\section{Concept of Check and Balance in Pakistan}
In Pakistan, checks and balances are implemented as follows:

\begin{itemize}
  \item Bicameral Legislature: Pakistan's legislature includes the National Assembly and the Senate, with more legislative power held by the National Assembly.
  \item Separation of Powers: The government is divided into the Legislative, Executive, and Judicial branches.
  \item Legislative Branch: Responsible for creating laws, composed of members elected through public elections.
  \item Executive Branch: Enforces and implements laws, with real authority held by the Prime Minister and Cabinet.
  \item Judicial Branch: Ensures laws align with the constitution and can declare them unconstitutional.
  \item Executive Oversight: The Prime Minister is accountable to the National Assembly and Senate, removable through a vote of no confidence.
  \item Legislative Control: The President can veto bills, but both houses can override with a simple majority vote.
  \item Judicial Review: The judiciary reviews laws and government actions for constitutional compliance.
  \item Election Commission: Conducts free and fair elections, maintaining the democratic process for the legislative branch.
\end{itemize}

\end{document}
